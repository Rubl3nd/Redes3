\documentclass[12pt]{article}
\oddsidemargin 0in
\textwidth 6.75in
\topmargin 0pt
\headheight 0in
\textheight 8.5in
\usepackage{graphicx}
\usepackage{listings}
\usepackage{amsmath}
\usepackage{amsfonts}
\title{Reporte de  Practica 1}
\author{Navarro P'erez Rafael \\Salcedo Barron Ruben\\
		Escuela Superior de C\'omputo\\
		ADMINISTRACION DE SERVICIOS EN RED\\ 
		4CM3
		}
\date{1 de marzo del 2019}

\begin{document}
\maketitle
\newpage 
\tableofcontents
\newpage 
\section{Primera parte}
\paragraph*{Introducción}
EL SNMP (Simple Network Management Protocol) es el protocolo más utilizado para la gestión de
redes IP basadas en internet. La versión original, ahora conocido como SNMPv1, es ampliamente
difundida. SNMPv2 añade funcionalidad a la versión original, pero no se ocupa de sus limitaciones
de seguridad; esta norma relativamente reciente no ha alcanzado mucha aceptación. La versión
SNMPv3 que conserva las mejoras funcionales de SNMPv2 y añade potentes funciones de
privacidad y autenticación.\\
El protocolo simple de administración de red (SNMP), publicado en 1988, fue diseñado para
proporcionar una implementación sencilla, así como facilitar el trabajo de gestión de redes de
múltiples proveedores (enrutadores, servidores, estaciones de trabajo y otros recursos de la red).
La especificación de SNMP tiene como objetivo:\\\\
• Definir un protocolo para el intercambio de información entre uno o más sistemas de
gestión y un número de agentes\\
• Proporcionar un marco para dar formato y almacenamiento de información de gestión\\
• Define una serie de variables de información de gestión de propósito general, u objetos\\
\\La versión original de SNMP (ahora conocido como SNMPv1) se convirtió rápidamente en el
esquema de gestión de la red más utilizado. Sin embargo, como el uso del protocolo se generalizó,
se hicieron evidentes sus deficiencias. Estas incluyen la falta de comunicación-manager-manager, la
incapacidad para hacer la transferencia de datos a granel, y la falta de seguridad. \\
SNMPv2 no ha recibido la aceptación que sus diseñadores anticiparon. Mientras que las mejoras
funcionales han sido bienvenidas, los desarrolladores encontraron las modificaciones de seguridad
para SNMPv2 demasiado complejas. En consecuencia, el grupo de trabajo SNMPv2 se reactivó para
proporcionar una mejora de los documentos SNMPv2.\\
El resultado de este esfuerzo ha sido un éxito menor y un gran fracaso. El éxito de menor
importancia es la mejora de los aspectos funcionales de SNMPv2. El gran fracaso radica en el área
de la seguridad. El grupo de trabajo fue incapaz de resolver el problema, y surgieron dos enfoques
diferentes. Con esta mejora, la parte funcional de SNMPv2 progresó de una propuesta a un estándar
de Internet a partir de 1996. Luego, en 1997, empezó a trabajar en SNMPv3, lo que hace cambios
funcionales menores e incorpora un nuevo enfoque de seguridad.
\subsection{Capturas}
\section{Segunda parte}
\subsection{Capturas}
\section{Tercera parte}
\subsection{Preguntas}
\subsection{Capturas}
\subsection{Código}
\lstset{language=C, breaklines=true, basicstyle=\footnotesize}
\begin{lstlisting}[frame=single]
#include <stdio.h>
#include <math.h>
#include<time.h> 
unsigned long generar(int x);
int main(){
            int resp,r,rep=0;;
			do
			{
			int num=0;
			printf("1.-Manual \n 2.-Automatico\n");
    
\end{lstlisting}
\end{document}