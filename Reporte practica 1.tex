\documentclass[12pt]{article}
\oddsidemargin 0in
\textwidth 6.75in
\topmargin 0pt
\headheight 0in
\textheight 8.5in
\usepackage{graphicx}
\usepackage{listings}
\usepackage{amsmath}
\usepackage{amsfonts}
\title{Reporte de  Programas\\1er Parcial}
\author{Salcedo Barron Ruben\\
		Escuela Superior de C\'omputo\\
		Teor\'ia Computacional\\ 
		2CM5
		}
\date{5 de septiembre del 2016}

\begin{document}
\maketitle
\tableofcontents
\newpage 
\section{Universo Binario}

Programa que genera todo el universo del alfabeto binario, dada $n$ que ser\'a la potencia tope que imprime todo el alfabeto.
\subsection{Manual}

\subsection{Codigo}
\lstset{language=C, breaklines=true, basicstyle=\footnotesize}
\begin{lstlisting}[frame=single]
#include <stdio.h>
#include <math.h>
#include<time.h> 
unsigned long generar(int x);
int main(){
            int resp,r,rep=0;;
			do
			{
			int num=0;
			printf("1.-Manual \n 2.-Automatico\n");
             scanf("%d",&resp);
			 switch(resp)
			 {
             case 1: 
             printf("Ingrese n :\n");
             scanf("%d",&num);
             generar(num);
		    printf("Desea repetir \n 1-NO\n 2-SI\n");
			scanf("%d",&rep);
			 break;
			 case 2: 
             srand(time(NULL));
			  r=rand()%20;
			 printf("numero es %d",r);
           num=r;
		   generar(num);
		    printf(" Repetir \n 1-NO\n 2-SI \n");
			rep=1+rand()%2;
			printf(" %d ",rep);
		   break;
		   default: printf("opcion incorrecta "); break;
		   }
		   }while(rep!=1);
		 return 0;
} 
unsigned long generar (int x)
{
FILE *doc;
doc=fopen("universo.txt","w+");
 unsigned long y,n=0,i,j;
fprintf(doc,"{e,"); 
    for(i=0;i<pow(2,x);i++){
          y=i;  
		while(y!=0){
            fprintf(doc,"%d",y%2);
                y=y/2;
                n++;
					}
               for(j=n;j<x;j++){
                 fprintf(doc,"0");
					}           
			fprintf(doc,",");
		}
		fprintf(doc,"}");
	printf("\n");
}
\end{lstlisting}

\end{document}