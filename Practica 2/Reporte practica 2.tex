\documentclass[12pt]{article}
\oddsidemargin 0in
\textwidth 6.75in
\topmargin 0pt
\headheight 0in
\textheight 8.5in
\usepackage{graphicx}
\usepackage{listings}
\usepackage{amsmath}
\usepackage{amsfonts}
\title{Reporte de  Practica 2}
\author{Navarro P'erez Rafael \\Salcedo Barron Ruben\\ Castillo Torres Alma Laura\\
		Escuela Superior de C\'omputo\\
		ADMINISTRACION DE SERVICIOS EN RED\\ 
		4CM3
		}
\date{4 de abril del 2019}

\begin{document}
\maketitle
\newpage 
\tableofcontents
\newpage 
\section{Primera parte}
\paragraph*{Introducción Método de mínimos Cuadrados}
Una línea del mejor ajuste es una recta que muestra la mejor aproximación del
conjunto dado de datos dispersos. Se utiliza para estudiar la naturaleza de la
relación entre dos variables.
Una línea del mejor ajuste puede determinarse usando un método de “simple vista”
dibujando una línea recta en un diagrama de dispersión de modo que el número de
puntos por encima y por debajo de la línea sea aproximadamente igual.
Una forma más precisa de encontrar la línea del mejor ajuste es el método de
mínimos cuadrdados.
\subsection{Capturas}
\section{Segunda parte}
\subsection{Capturas}
\end{document}